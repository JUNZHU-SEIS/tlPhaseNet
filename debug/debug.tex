\documentclass{article}
\title{\textbf{Debug history}}
\author{\textbf{Jun ZHU}}
%\date{March 25, 2021}
\usepackage[normalem]{ulem}
\usepackage{hyperref}
\hypersetup{
    colorlinks=true,
    linkcolor=blue,
    filecolor=magenta,      
    urlcolor=cyan,
    pdftitle={Overleaf Example},
    pdfpagemode=FullScreen,
    }
\usepackage{graphicx}

\usepackage{float}
\graphicspath{{image/}}
\usepackage{indentfirst}
\setlength{\parindent}{2em}
\usepackage{listings}
\usepackage{color}
\definecolor{dkgreen}{rgb}{0,0.6,0}
\definecolor{gray}{rgb}{0.5,0.5,0.5}
\definecolor{mauve}{rgb}{0.58,0,0.82}
\lstset{frame=tb,
  language=Python,
  aboveskip=3mm,
  belowskip=3mm,
  showstringspaces=false,
  columns=flexible,
  basicstyle={\small\ttfamily},
  numbers=none,
  numberstyle=\tiny\color{gray},
  keywordstyle=\color{blue},
  commentstyle=\color{dkgreen},
  stringstyle=\color{mauve},
  breaklines=true,
  breakatwhitespace=true,
  tabsize=3
}
\begin{document}
\maketitle
\section{2021/07/08}
\par{\textbf{project description:}}
\begin{itemize}
  \item phase picking file path: /home/as/Data/tlPhasenet/eqphas/2014\_out\_dist400.txt.ok4Li
  \item phase picking file: \# \$(origin time in UTC+8 \ldots) \{\$(station code + network code) \ldots \ldots \$(phase name)   \ldots \$(pick time in UTC+8) \ldots \ldots \$(channel code)\}
  \item waveform file path: /home/as/Data/tlPhasenet/xichang/20140102.002.155020.650/SC.GZA\ldots
  \item 29497 files span from Jan 2014 to Dec 2019
  \item target input file format: csv
  \item target imput file example: fname,itp,its,channels \{\$(npzfile name),\$(location of P-),\$(location of S-),\$(involved channel(s), like EHE\_EHN\_EHZ)\}
\end{itemize}

\par{\textbf{preprocess description:}}
\begin{itemize}
  \item remove empty folders inside /home/as/Data/tlPhasenet/xichang/
  \item \textcolor{red}{only the phase picking record involving both P- and S- picks reserved}
  \item \textit {discard the repeating P- and S- picks}
  \item \textit {some stream are not identically long in terms of each trace} (common segment of 3-components)
  \item \textit {some picks repeat in the file, just conserve the first record, use the shell script named \textbf{removerepeat.sh}}
  \item \textit {some picks have over 3 traces in the folder, such as oritention code 000, 001, etc. What's worse, their sample rates are not the same. I just ignore them.}
  \item {some picks are not involved in the 90-second training waveform segment, some are not involved in the 30-second test waveform segment. I just use the recording which involves both P- and S- picks in both train \& test segment.}
  \item \underline {\textcolor{red}{screen the picks \& divide the datasets based on the seismic events}}
  \item \underline {\textcolor{red}{filter the waveform (the approach adopted by Chenping Chai et al.)}}
\end{itemize}

\par{\textbf{preprocess result overview: \sout{54993} 52977 recordings (17919 events) left}}\\
a sample waveform
\begin{figure}[H]
  \centering
  \includegraphics{samplewaveform.png}
%  \caption{}
  \label{samplewaveform}
\end{figure}


\section{2021/07/12}
\par{fix the problem of preprocessing: can't cast dtype('float32') to dtype('int32')}
\begin{lstlisting}
def normaliz(self, data):
	data = data - np.mean(data, axis=0, keepdims=True)
	# omit below
	return
\end{lstlisting}


\section{2021/07/15}
\par{evaluation metrics:}
\begin{itemize}
  \item peak probabilities above 0.5 should be counted as positive picks
  \item $\Delta t<0.1$s is true positives, picks with larger residuals are counted as false positives
  \item mean and standard deviation are calculated on residuals ($\Delta t<0.5s$)
  \item precision; recall; F1; mean; standard deviation
\end{itemize}

\par{PhaseNet configuration:}
\begin{itemize}
  \item 779514 recordings
  \item different types of instruments; a wide range of SNR
  \item 30-s time window; 3001 points
  \item \textcolor{red}{varied position of the arrivals within the window}
  \item \textcolor{red}{100 Hz}
  \item remove mean and divide by std
\end{itemize}

\section{2021/07/16}
\par{\textcolor{red}{Whether or not keep the recordings whose S- pick is outside the window?} No.}
\par{How to freeze and load model in tensoeflow? see \href{https://blog.metaflow.fr/tensorflow-how-to-freeze-a-model-and-serve-it-with-a-python-api-d4f3596b3adc}{blog} here.}
\section{2021/07/21}
\par{Train history of the re-train PhaseNet:}
\begin{figure}[H]
  \centering
  \includegraphics[scale=0.8]{retrainLoss.png}
  \caption{re-train model}
  \label{fig:<+label+>}
\end{figure}

\par{Train history of the raw PhaseNet:}
\begin{figure}[H]
  \centering
  \includegraphics[scale=0.8]{rawLoss.png}
  \caption{raw model}
  \label{fig:<+label+>}
\end{figure}

\par{For pretrain model, its loss rises unexpectedly. It could be attributed to the utilization of the high learning rate. No improvement is observed even though modifying the learning rate.}
\begin{figure}[H]
  \centering
  \includegraphics[scale=0.8]{pretrainLoss.png}
  \caption{pretrain model}
  \label{fig:<+label+>}
\end{figure}

\section{2021/07/29}
\par{Some problems of the pick:}
\begin{itemize}
  \item Some P- picks are mistaken as S- picks. \textbf{GZ WNT 20140502111205}
\end{itemize}
\section{Reference}


\end{document}
